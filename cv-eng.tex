%% start of file `template.tex'.
%% Copyright 2006-2015 Xavier Danaux (xdanaux@gmail.com).
%
% This work may be distributed and/or modified under the
% conditions of the LaTeX Project Public License version 1.3c,
% available at http://www.latex-project.org/lppl/.


\documentclass[10pt,a4paper,roman]{moderncv}        % possible options include font size ('10pt', '11pt' and '12pt'), paper size ('a4paper', 'letterpaper', 'a5paper', 'legalpaper', 'executivepaper' and 'landscape') and font family ('sans' and 'roman')

% moderncv themes
\moderncvstyle{banking}                             % style options are 'casual' (default), 'classic', 'banking', 'oldstyle' and 'fancy'
\moderncvcolor{burgundy}                               % color options 'black', 'blue' (default), 'burgundy', 'green', 'grey', 'orange', 'purple' and 'red'
%\renewcommand{\familydefault}{\sfdefault}         % to set the default font; use '\sfdefault' for the default sans serif font, '\rmdefault' for the default roman one, or any tex font name
%\nopagenumbers{}                                  % uncomment to suppress automatic page numbering for CVs longer than one page
% character encoding
%\usepackage[utf8]{inputenc}                       % if you are not using xelatex ou lualatex, replace by the encoding you are using
%\usepackage{CJKutf8}                              % if you need to use CJK to typeset your resume in Chinese, Japanese or Korean

% adjust the page margins
\usepackage[scale=0.75]{geometry}
%\setlength{\hintscolumnwidth}{3cm}                % if you want to change the width of the column with the dates
%\setlength{\makecvheadnamewidth}{10cm}            % for the 'classic' style, if you want to force the width allocated to your name and avoid line breaks. be careful though, the length is normally calculated to avoid any overlap with your personal info; use this at your own typographical risks...
% setting the colours according to \moderncvcolor
\colorlet{languagecolor}{color1}
\colorlet{nolanguagecolor}{color2}
\newcount\languagecount
\newcommand\languageknowledge[2]
  {%
    % if you change the 4cm you can change the distance, the 4cm is the space
    % reserved for the language's name, immediately after it the dots are
    % printed.
    %\hfill % if you want the dots to be at the right margin
    \languagecount=0
    \loop\ifnum\languagecount<#2
      \advance\languagecount1
      \textcolor{languagecolor}{$\bullet$}%
    \repeat
    \loop\ifnum\languagecount<5
      \advance\languagecount1
      \textcolor{nolanguagecolor}{$\bullet$}%
    \repeat
    \makebox[3cm][l]{ \textbf{#1}}%
  }

% personal data
\name{Gianluca}{Aguzzi}
\title{\\Postdoctoral Researcher}
\address{Via Mulini 23/25}{47521}{Italy, Cesena}
\email{gianluca.aguzzi@unibo.it}
\homepage{cric96.github.io/}
\social[orcid]{0000-0002-1553-4561}
\social[github]{cric96}
\social[stackoverflow]{gianluca-aguzzi}
% bibliography adjustements (only useful if you make citations in your resume, or print a list of publications using BibTeX)
%   to show numerical labels in the bibliography (default is to show no labels)
%\makeatletter\renewcommand*{\bibliographyitemlabel}{\@biblabel{\arabic{enumiv}}}\makeatother
\renewcommand*{\bibliographyitemlabel}{[\arabic{enumiv}]}
%   to redefine the bibliography heading string ("Publications")
%\renewcommand{\refname}{Articles}

% bibliography with mutiple entries
\usepackage{multibib}
\newcites{journals,conferences}{{Journals},{Conference, Workshops, and Chapters}}
%----------------------------------------------------------------------------------
%            content
%----------------------------------------------------------------------------------
\begin{document}

\makecvtitle

\subsection{Current Placement}
\cventry{2023 November -- 2025 November }{Postdoctoral Fellow (``assegno di ricerca'')}%
{Alma Mater Studiorum -- University of Bologna}{Cesena}
{}{Currently appointed as a Postdoctoral Research Fellow in the COMMONS-WEARS project (funded as a PRIN), focusing on the engineering collective applications in complex, layered environments with multi-mobile edge computing architectures.}
\subsection{Research Profile}
\cvitem{}{
My research sits at the intersection of \emph{collective adaptive systems engineering} (software engineering) and \emph{machine learning} methodologies for distributed system. Particularly, my work contributes to several specific areas:
\begin{itemize}
  \item \textbf{Software Engineering for Collective Behaviors:} In this research area, I focus on advancing the macroprogramming paradigm (programming systems from a global perspective) in the context of very large distributed systems (e.g., swarm robotics, smart cities). 
  I propose novel solutions for managing groups of robots in a distributed way based on spatial computing principles~\citejournals{aguzzi2023field-sensing}, 
  along with new programming approaches through frameworks like Macroswarm~\citejournals{aguzzi2025macroswarm}. 
  My work includes fundamental research on how to engineer such applications, 
  leading to foundational publications on collective autonomy~\citejournals{casadei2021programming} and comprehensive roadmaps for future development~\cite{casadei2024software}. 
  This research covers several aspects of the programming pipeline, 
  including runtime monitoring for system safety~\citeconferences{aguzzi2024optimising}, 
  programming frameworks like ScaFi~\cite{casadei2022scafi} and Macroswarm~\cite{aguzzi2023macroswarm} and others~\cite{aguzzi2024scafi,aguzzi2021scafi}, 
  novel reactive models for swarm programming~\cite{casadei2023self},
  and the development of novel architectural patterns for collective computation~\cite{aguzzi2022dynamic}.
  \item \textbf{Hybrid Methodologies for Collective Intelligence:} In this research area, building on the advancements from my work on collective behaviors, 
  I integrate machine learning solutions to improve currently manual design approaches. 
  The goal is to both enhance system adaptability (through learning) 
  and improve overall efficiency. 
  This integration was first outlined through comprehensive roadmaps~\cite{aguzzi2022machine,aguzzi2021research}, 
  then implemented in several ways: 
  leveraging Multi-Agent Reinforcement Learning (MARL) for program synthesis~\cite{aguzzi2022towards}, 
  improving macroprogramming execution with scheduling learned via MARL~\cite{aguzzi2022addressing}, 
  and enhancing current MARL solutions for swarms by using macroprogramming as a way to represent agent state~\cite{aguzzi2023field}.
  \item \textbf{Advanced Methodologies for Cooperative and Scalable Learning:} 
  Based on the insights gained from programming scalable systems with macroprogramming, 
  I bring these advancements to \emph{cooperative learning}, 
  highlighting the need for scalable solutions. 
  In this regard, 
  I propose novel MARL approaches based on neighborhood policies~\cite{DBLP:conf/sac/MalucelliDAV25} 
  and explore new federated learning solutions that avoid central points of failure~\cite{domini2024field,domini2024proximity} (work that led to a Marie Curie project with an associated Seal of Excellence award).
  In this area, I also proposed novel framework for cooperative many agent deep reinforcement learning in Scala~\cite{domini2023scarlib} and novel simulation pipeline for large-scale systems~\cite{DBLP:conf/dsrt/DominiAPV24}.
  \item \textbf{Generative AI for Modern Applications:} 
  This area represents a natural continuation of my previous work, 
  particularly focusing on how to use generative AI for designing modern distributed applications.
   Initial work explores how to integrate small LLMs in chatbots for healthcare management~\cite{tschope2025novel,magnini2025open}. During this period, I've also supervised several theses on code generation with LLMs for domain-specific languages, 
   with the aim of integrating these advancements with macroprogramming paradigms to further enhance collective system development.
\end{itemize}
}
\cvitem{Bibliometrics}{}{
  \begin{itemize}
    \item \textbf{H-index:} 11 (Google Scholar), 9 (Scopus)
    \item \textbf{Citations:} 271 (Google Scholar), 184 (Scopus)
  \end{itemize}
}
\section{Education}

%%%%%
\cventry{2020--2023}{PhD in  Computer Science and Software Engineering}%
 {Alma Mater Studiorum -- University of Bologna}{Cesena}
 {\textit{with distinction}}{
My doctoral research centered on the design and engineering of large-scale systems, leveraging aggregate computing and advanced machine learning methods. Specifically, I investigated the integration of multi-agent reinforcement learning within cyber-pysical swarms--complex systems comprising numerous interacting agents operating in dynamic environments.
 }
\cvitem{}{Thesis: \emph{A language-based software engineering approach for cyber-physical swarms}}
\cvitem{}{Supervisors: \emph{Mirko Viroli}}
\cventry{2018--2020}{Master in  Computer Science and Software Engineering}%
 {Alma Mater Studiorum -- University of Bologna}{Cesena}
 {\textit{110 cum Laude}}{
  In this master I focused on the study of programming languages, software engineering, and distributed systems. I also explored the application of aggregate computing in the development of large-scale systems.
  Moreover, I developed a strong interest in the application of machine learning algorithms in the context of distributed systems, particularly in the field of multi-agent reinforcement learning.
 }
\cvitem{}{Awards: \emph{Best Master Thesis, Ca Foscari Award}}{}
\cvitem{}{Thesis: \emph{Scafi web: a Scala-JavaScript platform for executing, simulating, and controlling aggregate computing systems}}
\cvitem{}{Supervisors: \emph{Mirko Viroli, Roberto Casadei}}
%%%%
\cventry{2015--2018}{Bachelor in Computer Science and Software Engineering}{Alma Mater Studiorum -- University of Bologna}{Cesena}{\textit{110 cum Laude}}{}
\cvitem{}{Awards: \emph{Prize for Meritous Students}}
\cvitem{}{Thesis: \emph{Sviluppo di un front-end di simulazione per applicazioni aggregate nel framework Scafi}}
\cvitem{}{Supervisors: \emph{Mirko Viroli, Roberto Casadei}}

\cventry{2015--2018}{High School on Computer Science}{ITIS E. Mattei.}{Urbino}{\textit{100}}{}

\section{Publications}
Note: for Quartile (Q1-Q4), I took the highest quartile of the journal in which the paper is published in a range from the year it is accepted to the year of publication (e.g., if a paper is accepted in 2023, I took the max quartile of 2022 - 2023).
\nocitejournals{*}
\bibliographystylejournals{plainyr-rev}
\bibliographyjournals{journals}

\nociteconferences{*}
\bibliographystyleconferences{plainyr-rev}
\bibliographyconferences{conf}
\section{Scientific Activities}
%%%%%%% 
\subsection{Service in International Conferences}

\cventry{2025}{International Conference on Integrated Formal Method}{Artefact Evaluation Commitee}{}{}{}
\cventry{2025}{Workshop on Object to Agents -- WoA}{Session Chair}{}{}{}{}
\cventry{2025}{Workshop on Hot Topics in Distributed Machine Learning}{Program Chair Committe}{}{}{}
\cventry{2025}{LoStaN Workshop}{Program Chair Committee}{}{}{}
\cventry{2025}{Autonomic Computing and Self-Organizing Systems -- ACSOS}{Poster and Demo Session Organiser}{}{}{}
\cventry{2025}{Autonomic Computing and Self-Organizing Systems -- ACSOS}{Poster and Demo Session Organiser}{}{}{}
\cventry{2025}{Workshop on DIStributed COLlective Intelligence -- DISCOLI}{Program Chair Committe}{}{}{}
\cventry{2025}{European Conference on Object-Oriented Programming -- ECOOP}{Artifact Evaluation Committee}{}{}{}
\cventry{2025}{Workshop on DTs ecosystems and Application -- Digita}{Program Chair Committe}{}{}{}
\cventry{2024}{Workshop on Neuro-Symbolic Software Engineering -- NSE}{Program Chair Committe}{}{}{}
\cventry{2024}{International Conference on Software Language Engineering -- SLE}{Artifact Evaluation Committee}{}{}{}
\cventry{2024}{Autonomic Computing and Self-Organizing Systems -- ACSOS}{Demo and Poster Committee}{}{}{}
\cventry{2024}{Workshop on DIStributed COLlective Intelligence -- DISCOLI}{Organising Chair Committe}{}{}{}
\cventry{2024}{Workshop on Medical Applications with DTs and Edge-cloud Continuum -- MADTECC}{Program Chair Committe}{}{}{}
\cventry{2023}{International Conference on Pervasive Computing and Communications -- PerCom}{Artifact Evaluation Committee}{}{}{}
\cventry{2023}{Workshop on DIStributed COLlective Intelligence -- DISCOLI}{Program Chair Committe}{}{}{}
\cventry{2022}{International Conference on Coordination Models and Languages -- DisCoTec}{Artifact Evaluation Committee}{}{}{}
\cventry{2021}{International Conference on Autonomic Computing and Self-Organizing Systems -- ACSOS}{Artifact Evaluation Committee}{}{}{}

\subsection{Service in International Journals}
\cventry{2025}{Managing Guest Editor}{MDPI's Applied Science (IF 2.5)}{}{}{Emerging Techniques in Engineering Intelligent Agents and Multi-Agent Systems \ \url{https://www.mdpi.com/journal/applsci/special_issues/36128M0RBR}}

\subsection{Presentations in International Conferences}
\cventry{}{\textbf{COORDINATION 2024}}{}{}{}{Scafi-blocks: A visual aggregate programming environment for low-code swarm design~\citeconferences{aguzzi2024scafi}}
\cventry{}{\textbf{DISCOLI 2024}}{}{}{}{Engineering distributed collective intelligence in cyber-
physical swarms~\citeconferences{aguzzi2024engineering}}
\cventry{}{\textbf{ACSOS 2023}}{}{}{}{Programming (and learning)
self-adaptive \& self-organising behaviour with scafi: for swarms, edge-cloud ecosystems, and more~\citeconferences{casadei2023programming}}
\cventry{}{\textbf{ACSOS 2023}}{}{}{}{Field-informed Reinforcement Learning of Collective Tasks with Graph Neural Networks~\citeconferences{aguzzi2023field}}
\cventry{}{\textbf{COORDINATION 2023}}{}{}{}{Macroswarm: A field-based compositional framework for swarm programming~\citeconferences{aguzzi2023macroswarm}}
\cventry{}{\textbf{COORDINATION 2023}}{}{}{}{Scarlib: A framework for cooperative many agent deep reinforcement learning in Scala~\citeconferences{domini2023scarlib}}
\cventry{}{\textbf{ACSOS 2022}}{}{}{}{Addressing Collective Computations Efficiency: Towards a Platform-level Reinforcement Learning Approach~\citeconferences{aguzzi2022addressing}}
\cventry{}{\textbf{DISCOLI 2023}}{}{}{}{Machine learning for aggregate computing: a research roadmap~\citeconferences{aguzzi2022machine}}
\cventry{}{\textbf{COORDINATION 2022}}{}{}{}{Towards reinforcement learning-based aggregate computing~\citeconferences{aguzzi2022towards}}
\cventry{}{\textbf{Doctoral Symposium International @ ACSOS 2021}}{}{}{}{Research directions for aggregate computing with machine learning~\citeconferences{aguzzi2021research}}
\cventry{}{\textbf{COORDINATION 2021}}{}{}{}{ScaFi-Web: A Web-Based Application for Field-Based Coordination Programming~\citeconferences{aguzzi2021scafi}}


\subsection{Awards}
\cventry{March 2025}{Seal of Excellence (Marie Curie)}{European Commission}{}{}{}
\cventry{September 2024}{Best Poster Award}{ACSOS 2024}{}{}{\emph{https://github.com/DanySK/poster-2024-acsos-imageonomics-drones}}
\cventry{November 2023}{Best Master Thesis }{Sergio Focardi Awards}{Thesis: }{}{Scafi web: a Scala-JavaScript platform for executing, simulating, and controlling aggregate computing systems\emph{https://www.serinar.unibo.it/gianluca-aguzzi-si-aggiudica-la-ii-edizione-del-premio-di-laurea-sergio-focardi/}}

\subsection{Research Grant}
\cventry{November 2024}{Gemma 2 Research Award}{Google}{}{}{Value: ~10k USD}
\subsection{Visiting}
\cventry{August 2023 - November 2023 (3 months)}{Visiting PhD}{Aarhus University -- Lukas Esterle}{Aarhus, Denmark}{}{
  During my visit abroad, I focused on applying graph neural networks to develop distributed controllers.
  This research culminated in the publication of the paper entitled ``Field-informed reinforcement learning of collective tasks with graph neural network''. 
  Furthermore, I continued collaboration on federated learning for large-scale systems, an effort that also led to the Marie Curie Seal of Excellence.
}

\subsection{Volunteering}
\cventry{2022}{International Conference on Distributed Computing Systems - \emph{ICDCS}}{Student Volunteer}{}{}{}
\cventry{2022}{Internation Conference on Autonomic Computing and Self-Organising Systems - ACSOS}{Student Volunteer}{}{}{}

\subsection{Review Activity}
\cventry{}{Journals}{}{}{}{
\begin{itemize}
\item Pervasive and Mobile Computing
\item Future Generation Computer Systems
\item MDPI Sensors
\item Science of Computer Programming
\item Scientific Programming
\item Frontiers in Robotics and AI
\item Autonomous Agents and Multi-Agent Systems
\item Transactions on Autonomous and Adaptive Systems
\item PeerJ Computer Science
\item Journal of Medical Systems
\end{itemize}
}

\cventry{}{Conferences}{}{}{}{
\begin{itemize}
\item COORDINATION (2023 - 2025)
\item ACSOS (2023 - 2025)
\item AAMAS (2023)
\item PerCom (2023)
\item SAC (2024)
\item ICAART (2023)
\end{itemize}
}

\cventry{}{Workshops}{}{}{}{
\begin{itemize}
\item ASE NIER (2023)
\item MADTECC (2024)
\item Digita (2023 - 2025)
\item DISCOLI (2022 - 2025)
\item AIxIA (2025)
\item NSE (2024)
\end{itemize}
}

\subsection{Research Group Collaboration}
\cventry{2021 - }{University of Bologna}{Prof. Mirko Viroli}{}{}{In Prof. Viroli's research group, my activities have mainly focused on the topics of aggregate computing and multi-agent reinforcement learning applied to cyber swarms systems.}
\cventry{2021 - }{University of Turin}{Prof. Ferruccio Damiani}{}{}{
  In Ferruccio Damini's group, our primary focus was on the application of aggregate computing in swarm robotics. This fruitful collaboration resulted in the publication of the paper titled ``A field-based computing approach for sensing-driven clustering in robot swarms.''}
\cventry{2021}{St. Gallen University, Switzerland}{Prof. Guido Salvaneschi}{}{}{In collaboration with Guido Salvaneschi, we endeavoured to expand the concepts of pulverized architecture through multitier programming languages. Our joint efforts culminated in the publication of the paper titled ``Towards Pulverized Architectures for Collective Adaptive Systems through Multi-tier Programming''}
\cventry{2022 -}{Aarhus Universitat, Denmark}{Prof. Lukas Esterle}{}{}{Throughout my time abroad, our research was centred around exploring distributed collective intelligence within the realm of large-scale systems. Our primary emphasis was on the application of graph neural networks for developing distributed controllers.}

\subsection{PhD Schools}
\cventry{2023}{PhD Summer School}{Bertinoro Summer School}{}{}{}
\cventry{2023}{PhD Summer School}{10$^{th}$ DeepLearn Summer School}{}{}{}
\cventry{2021}{PhD Summer School}{22$^{nd}$ European Agent Systems Summer School}{}{}{}


\section{Teaching}
\subsection{Courses}
\cventry{July 2025}{PhD School}{Aggregate Programming for Internet Of Things -- 12 hours}{}{Turin}{
In this course I teach some advanced stuff in aggregate computing, which implies understanding 
the notion of time independence, spatial algorithms (gradient, data collection) and 
introduction for runtime verification.
This course was co-organized with the University of Turin with the supervision of Damiani Ferruccio
}
\cventry{April 2025}{Bachelor in Computer Science}{Compose in Android Development -- 10 hours (teaching module)}%
{HVL, Norway}{}{
  In this course, I teach the Compose library in Android development, focusing on its declarative programming model and how it simplifies UI development. The curriculum covers the fundamentals of Compose, including composable functions, state management, and UI design principles. Students engage in hands-on projects to apply their knowledge in real-world Android applications.
  }{}

\cventry{2024 - 2025}{Laurea Magistrale in Ingegneria e Scienze Informatiche}{Advanced Software Design and Modelling -- 20 hours (teaching module)}{University of Bologna}{}{
    In this advanced course, I lead modules on integrating generative AI into software engineering practices, with particular emphasis on leveraging large language models for automated code generation, design pattern implementation, and technical documentation. Additionally, I teach cutting-edge reinforcement learning concepts with focus on multi-agent systems and distributed intelligence architectures, preparing students for applying these technologies in complex software environments.
      }{}
\cventry{2024 - 2025}{\small Laurea Professionalizzante in Tecnologie dei Sistemi Informatici}{Software Design and Development -- 60 hours (in charge)}{University of Bologna}{}{
    In this course, I teach fundamental software design and development principles, emphasizing object-oriented programming concepts, design patterns, and agile methodologies. The curriculum covers software architecture, testing strategies, and best practices for building maintainable and scalable applications. Students engage in hands-on projects to apply theoretical knowledge to real-world software development challenges.
    }{}
\cventry{2023 - 2024}{{\small Laurea Professionalizzante in Tecnologie dei Sistemi Informatici}}{Software Design and Development -- 30 hours (teaching module)}{University of Bologna}{}{
    In this course, I teach fundamental software design and development principles, emphasizing object-oriented programming concepts, design patterns, and agile methodologies. The curriculum covers software architecture, testing strategies, and best practices for building maintainable and scalable applications. Students engage in hands-on projects to apply theoretical knowledge to real-world software development challenges.
    }{}
\cventry{July 2023}{Introduction to Reinforcement Learning -- 4 hours}%
{Advanced School in Artificial Intelligence}{\url{https://asai-er.github.io/services/docenti/}}
 {}{}

\subsection{Tutoring}
\cventry{2022--today}{Master in Computer Science and Engineering -- 30 hours}%
 {Concurrent and Distributed Programming}{University of Bologna}
 {}{}
\cventry{2022--today}{Master in Computer Science and Engineering -- 30 hours}%
 {Programming and Development Paradigms}{University of Bologna}
 {}{}

\cventry{2018 - 2019}{Snap! courses}{CRIAD Coding}{Grade schools}{}{}
\subsection{Thesis (Co)Supervisor - Selected}
\cventry{2024}{Master Thesis}{}{}{Student: Luce Deluigi}{\textbf{Design and implementation of a scalable domain specific language foundation for ScaFi with Scala 3.}}
\cventry{2024}{Master Thesis}{}{}{Student: Davide Domini}{\textbf{Aggregate Computing and Many-Agent Reinforcement Learning: Towards a Hybrid Toolchain}}
\cventry{2023}{Master Thesis}{}{}{Student: Francesco Dente}{\textbf{A functional-reactive perspective on the Aggregate Computing paradigm}}
\cventry{2023}{Master Thesis}{}{}{Student: Giacomo Cavalieri}{\textbf{Gestione degli effetti in linguaggi di programmazione funzionale: tecniche di modellazione e interpretazione}}
\cventry{2022}{Bachelor Thesis}{}{}{Student: Cerioni, Matteo}{\textbf{Progettazione di un ambiente di programmazione visuale block-based per ScaFi.}}
\cventry{2022}{Bachelor Thesis}{}{}{Student: Mancini, Kevin}{\textbf{ScaFi: Integration and Performance Analysis with Scala Native.}}
\cventry{}{}{}{}{}{For a complete list of supervised theses (over 20 thesis), please visit AMS thesis: \url{https://amslaurea.unibo.it/view/relatore/Aguzzi=3AGianluca=3A=3A/}}{}
\subsection{Invited Talks}
\cventry{2025}{\textbf{Autonomous Agents with Language Chains -- 1 hour}}{}{}{}{Mini School Workshop on Object to Agents -- WoA}
\cventry{2025}{Seminar: \textbf{Introduction of LangChain -- 1 hour}}{}{}{}{Deep Learning - University of Urbino}
\cventry{2024}{Seminar: \textbf{Multi-Agent Reinforcement Learning - Introduction -- 3 hours}}{}{}{}{Advanced Software Modelling and Design}
\cventry{2024}{Seminar: \textbf{Deep Reinforcement Learning -- Introduction -- 1 hour}}{}{}{}
{Fundamentals of Artifical Intellingence - University of Urbino}
\cventry{2024}{Seminar: \textbf{It's all about effects - Effect systems in Functional Programming -- 2 hour}}{}{}{}{Advanced Software Modelling and Design}
\cventry{2024}{Seminar: \textbf{Leveraging Large Language Models in Software Engineering -- 2 hours}}{}{}{}{Advanced Software Modelling and Design}
\cventry{2023}{Mini School: \textbf{Multi-Agent Reinforcement Learning, Unleashing Collective Intelligence -- 3 hours}}{}{}{}{Advanced School in Artificial Intelligence Summer School}
\cventry{2023}{Seminar: \textbf{Intro to Deep Reinforcement Learning -- 2 hours}}{}{}{}
{Fundamentals of Artificial Intelligence - University of Urbino}
\cventry{2022}{Talk: \textbf{Engineering Cyber-Physical Swarm -- 1 hours}}{}{}{}
{DIGIT lunch meetings -- Aarhus Universitat}%
\cventry{2022}{Seminar: \textbf{Multi-Agent Reinforcement Learning, Introduction -- 2 hours}}{}{}{}{Pervasive Computing - University of Bologna}%

\cventry{2022}{Seminar. \textbf{Scala to the large -- 2 hours}}{}{}{}%
{Programming and Development Paradigms - University of Bologna}

\cventry{2022}{Seminar: \textbf{Cross Platform in Scala -- 1 hour}}{}{}{}
{Programming and Development Paradigms - University of Bologna}

\cventry{2021}{Seminar: \textbf{On Collective Reinforcement Learning -- 2 hours}}{}{}{}
{Alma Mater Studiorum -- University of Bologna}{Talk @ Pervasive Computing}

\cventry{2021}{Seminar: \textbf{MVC meets Monad -- 1 hour}}{}{}{}
{Programming and Development Paradigms -- University of Bologna}

\cventry{2019}{Seminar: \textbf{Create your own video game in Snap!}}{}{}{}
{Orientation Fair - Forlì}

%\subsection{Student Supervisor}

\section{Software Projects}
\cventry{2025 -- today}{It is a framework which implemented a ``python'' version of Aggregate computing.}{Designer of Phyelds}{}{}{\url{https://github.com/phyelds/phyelds}}
\cventry{2023 -- today}{It is a field-based compositional framework for swarm programming.}{Designer of Macroswarm~\cite{aguzzi2025macroswarm}}{}{}{\url{https://github.com/scafi/macro-swarm}}
\cventry{2023 -- today}{It is a framework for reactive self-organizing programming}{CO-designer of FRASP~\cite{casadei2023self}}{}{}{\url{https://github.com/cric96/distributed-frp}}
\cventry{2023 -- today}{It is a framework for cooperative many agent deep reinforcement learning in Scala}{Designer of Scarlib~\cite{domini2024scarlib}}{}{}{\url{https://github.com/ScaRLib-group/ScaRLib}}
\cventry{2021 -- today}{It is a web-based application allowing in-browser editing and execution of ScaFi programs.}{Co-designer and main contributor of ScaFi-Web~\cite{aguzzi2021scafi}}{}{}{\url{https://github.com/scafi/scafi-web}}
\cventry{2021 -- today}{It is a Scala facade that enable the usage of open ai gyms in the JVM!}{Designer of scalapy-gym}{}{}{\url{https://github.com/cric96/scalapy-gym}}
%\cventry{2020 -- today}{Co-designer of Fluvium}{An IoT project for river controll that uses AWS lambda}{}{}{\url{https://github.com/sbricco-house/fluvium}}
\subsection{Open Source Contributions}
\cventry{2018 -- today}{Development of GUI \& simulator for ScaFi}{}{}{}{\url{https://github.com/scafi/scafi}}
\cventry{2021 -- today}{Contributions to ScaFi incarnations in Alchemist}{}{}{}{\url{https://github.com/AlchemistSimulator/Alchemist}}

\section{Professional Experience}
\cventry{2025 -- Today}{Software Engineering Consultant}{SOILMEC S.p.A.}{Cesena, Italy}{}{Providing strategic technical consulting focused on legacy system modernization and architectural improvements to enhance code maintainability and development efficiency.}
\cventry{2024 -- Today}{Software Engineering Consultant}{Flash Start}{Cesena, Italy}{}{Delivering specialized consulting services in software architecture redesign and implementation of modern development practices to optimize existing systems.}
\section{Technical Skills}

\subsection{Programming Languages}
\begin{cvcolumns}
  \cvcolumn{}{
    \cvitem{\languageknowledge{Scala}{5}}{}
    \cvitem{\languageknowledge{Java}{4}}{}
    \cvitem{\languageknowledge{C\#}{3}}{}
    \cvitem{\languageknowledge{C}{2}}{}
  }
  \cvcolumn{}{
    \cvitem{\languageknowledge{Kotlin}{3}}{}
    \cvitem{\languageknowledge{JavaScript}{3}}{}
    \cvitem{\languageknowledge{Bash}{2}}{}
    \cvitem{\languageknowledge{Prolog}{1}}{}
  }
  \cvcolumn{}{
    \cvitem{\languageknowledge{TypeScript}{1}}{}
    \cvitem{\languageknowledge{Haskell}{1}}{}
    \cvitem{\languageknowledge{C++}{2}}{}
  }
\end{cvcolumns}

\subsection{Other Languages}
\begin{cvcolumns}
  \cvcolumn{}{
    \cvitem{\languageknowledge{HTML}{3}}{}
    \cvitem{\languageknowledge{XML}{2}}{}
    \cvitem{\languageknowledge{JSON}{3}}{}
    \cvitem{\languageknowledge{RDF}{1}}{}{}
    }
  \cvcolumn{}{
    \cvitem{\languageknowledge{Markdown}{3}}{}
    \cvitem{\languageknowledge{LaTeX}{4}}{}
    \cvitem{\languageknowledge{OWL}{1}}{}
  }
  \cvcolumn{}{
    \cvitem{\languageknowledge{SPARQL}{1}}{}
    \cvitem{\languageknowledge{YAML}{2}}{}
    \cvitem{\languageknowledge{SQL}{2}}{}
  }
\end{cvcolumns}
\subsection{Libraries}
\begin{cvcolumns}
  \cvcolumn{}{
    \cvitem{\languageknowledge{Scala.js}{3}}{}
    \cvitem{\languageknowledge{Tensorflow}{2}}{}
    \cvitem{\languageknowledge{Pytorch}{1}}{}
    \cvitem{\languageknowledge{OpenAI Gym}{2}}{}
    }
  \cvcolumn{}{
    \cvitem{\languageknowledge{Monix}{4}}{}
    \cvitem{\languageknowledge{Matplotlib}{2}}{}
    \cvitem{\languageknowledge{Akka}{2}}{}
  }
  \cvcolumn{}{
    \cvitem{\languageknowledge{ScalaPy}{3}}{}
    \cvitem{\languageknowledge{Cats}{3}}{}
    \cvitem{\languageknowledge{ZIO}{1}}{}{}
  }
\end{cvcolumns}
\subsection{Software Tools}
\begin{cvcolumns}
  \cvcolumn{}{
    \cvitem{\languageknowledge{Gimp}{2}}{}{}
    \cvitem{\languageknowledge{Git}{4}}{}{}
    \cvitem{\languageknowledge{GHA}{2}}{}{}
    \cvitem{\languageknowledge{Docker}{2}}{}{}
    }
  \cvcolumn{}{
    \cvitem{\languageknowledge{Inkscape}{2}}{}{}
    \cvitem{\languageknowledge{Blender}{2}}{}{}
    \cvitem{\languageknowledge{OWL}{1}}{}{}
    \cvitem{\languageknowledge{Kdenlive}{1}}{}{}
  }
  \cvcolumn{}{
    \cvitem{\languageknowledge{NPM}{1}}{}{}
    \cvitem{\languageknowledge{SBT}{4}}{}{}
    \cvitem{\languageknowledge{Hugo}{1}}{}{}
    \cvitem{\languageknowledge{Gradle}{2}}{}{}
  }
\end{cvcolumns}

\section{Miscellaneous}
\cventry{2025}{\textbf{Presentation at GENERARE}}{}{}{}{Delivered an engaging talk on Large Language Models, demystifying advanced concepts for a broader audience.}
\cventry{2024}{\textbf{Presenting Aggregate Computing @ Researcher Night}}{}{}{}{In that occasion I presented Aggregate Computing with physical robots highlighting the potential of the paradigm.}
\cventry{2023}{\textbf{Scala Italy 2023}}{}{}{}{Scala in machine learning scenario: a personal experience}
\cventry{2020}{\textbf{Student class representative @ Alma Mater Studiorum}}{}{}{}{}
\cventry{2018}{\textbf{Presenting Snap! @ Researcher Night}}{}{}{}{In that occasion I presented Snap! to the public as a tool for teaching the computational thinking to the youngest.}
%\cventry{2013-2015}{Student class representative @ ITIS}{}{}{}{}

%\section{References}

% Publications from a BibTeX file without multibib
%  for numerical labels: \renewcommand{\bibliographyitemlabel}{\@biblabel{\arabic{enumiv}}}% CONSIDER MERGING WITH PREAMBLE PART
%  to redefine the heading string ("Publications"): \renewcommand{\refname}{Articles}

% Publications from a BibTeX file using the multibib package
%\section{Publications}
%\nocitebook{book1,book2}
%\bibliographystylebook{plain}
%\bibliographybook{publications}                   % 'publications' is the name of a BibTeX file
%\nocitemisc{misc1,misc2,misc3}
%\bibliographystylemisc{plain}
%\bibliographymisc{publications}                   % 'publications' is the name of a BibTeX file

\clearpage

%\clearpage\end{CJK*}                              % if you are typesetting your resume in Chinese using CJK; the \clearpage is required for fancyhdr to work correctly with CJK, though it kills the page numbering by making \lastpage undefined
\end{document}


%% end of file `template.tex'.