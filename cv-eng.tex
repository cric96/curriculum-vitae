%% start of file `template.tex'.
%% Copyright 2006-2015 Xavier Danaux (xdanaux@gmail.com).
%
% This work may be distributed and/or modified under the
% conditions of the LaTeX Project Public License version 1.3c,
% available at http://www.latex-project.org/lppl/.


\documentclass[11pt,a4paper,roman]{moderncv}        % possible options include font size ('10pt', '11pt' and '12pt'), paper size ('a4paper', 'letterpaper', 'a5paper', 'legalpaper', 'executivepaper' and 'landscape') and font family ('sans' and 'roman')

% moderncv themes
\moderncvstyle{classic}                             % style options are 'casual' (default), 'classic', 'banking', 'oldstyle' and 'fancy'
\moderncvcolor{burgundy}                               % color options 'black', 'blue' (default), 'burgundy', 'green', 'grey', 'orange', 'purple' and 'red'
%\renewcommand{\familydefault}{\sfdefault}         % to set the default font; use '\sfdefault' for the default sans serif font, '\rmdefault' for the default roman one, or any tex font name
%\nopagenumbers{}                                  % uncomment to suppress automatic page numbering for CVs longer than one page
% character encoding
%\usepackage[utf8]{inputenc}                       % if you are not using xelatex ou lualatex, replace by the encoding you are using
%\usepackage{CJKutf8}                              % if you need to use CJK to typeset your resume in Chinese, Japanese or Korean

% adjust the page margins
\usepackage[scale=0.75]{geometry}
%\setlength{\hintscolumnwidth}{3cm}                % if you want to change the width of the column with the dates
%\setlength{\makecvheadnamewidth}{10cm}            % for the 'classic' style, if you want to force the width allocated to your name and avoid line breaks. be careful though, the length is normally calculated to avoid any overlap with your personal info; use this at your own typographical risks...
% setting the colours according to \moderncvcolor
\colorlet{languagecolor}{color1}
\colorlet{nolanguagecolor}{color2}
\newcount\languagecount
\newcommand\languageknowledge[2]
  {%
    % if you change the 4cm you can change the distance, the 4cm is the space
    % reserved for the language's name, immediately after it the dots are
    % printed.
    %\hfill % if you want the dots to be at the right margin
    \languagecount=0
    \loop\ifnum\languagecount<#2
      \advance\languagecount1
      \textcolor{languagecolor}{$\bullet$}%
    \repeat
    \loop\ifnum\languagecount<5
      \advance\languagecount1
      \textcolor{nolanguagecolor}{$\bullet$}%
    \repeat
    \makebox[3cm][l]{ \textbf{#1}}%
  }

% personal data
\name{Gianluca}{Aguzzi}
\title{PhD Student}
\address{Via Mulini 23/25}{47521}{Italy, Cesena}
\email{gianluca.aguzzi@unibo.it}
\homepage{https://cric96.github.io/}
\social[orcid]{0000-0002-1553-4561}
\social[github]{cric96}
\social[stackoverflow]{gianluca-aguzzi}
% bibliography adjustements (only useful if you make citations in your resume, or print a list of publications using BibTeX)
%   to show numerical labels in the bibliography (default is to show no labels)
%\makeatletter\renewcommand*{\bibliographyitemlabel}{\@biblabel{\arabic{enumiv}}}\makeatother
\renewcommand*{\bibliographyitemlabel}{[\arabic{enumiv}]}
%   to redefine the bibliography heading string ("Publications")
%\renewcommand{\refname}{Articles}

% bibliography with mutiple entries
\usepackage{multibib}
\newcites{book,misc}{{Books},{Others}}
%----------------------------------------------------------------------------------
%            content
%----------------------------------------------------------------------------------
\begin{document}

\makecvtitle

\subsection{Current Placement}
\cventry{2020--today}{PhD in Computer Science And Engineering}%
 {Alma Mater Studiorum -- Università di Bologna}{Cesena}
 {}{}
\subsection{Research Theme}
\cvitem{}{
  My current research topics concern large-scale system coordination and machine learning algorithm in distributed and multi-agent systems. 
  In particular, I am interested in \emph{engineering} self-adaptive systems by means of \emph{models}, \emph{programming languages} and \emph{disciplines}. 
  Furthermore, I investigate the use of machine learning algorithms -- and in particular Reinforcement Learning -- in support of that engineering process making the collective program more \emph{effective}, \emph{efficient} and \emph{smarter}.
}
\section{Education}

\cventry{2021}{Attending @ PhD Summer School}{22$^{nd}$ European Agent Systems Summer School}{}{}{}
%%%%%
\cventry{2018--2020}{Master in Software Engineering}%
 {Alma Mater Studiorum -- Università di Bologna}{Cesena}
 {\textit{110 cum Laude}}{}
\cvitem{}{Thesis: \emph{Scafi web: a Scala-JavaScript platform for executing, simulating, and controlling aggregate computing systems}}
\cvitem{}{Supervisors: \emph{Mirko Viroli, Roberto Casadei}}
%%%%
\cventry{2015--2018}{Bachelor in Computer Science and Software Engineering}{Alma Mater Studiorum -- Università di Bologna}{Cesena}{\textit{110 cum Laude}}{}
\cvitem{}{Thesis: \emph{Sviluppo di un front-end di simulazione per applicazioni aggregate nel framework Scafi}}
\cvitem{}{Supervisors: \emph{Mirko Viroli, Roberto Casadei}}

\cventry{2015--2018}{High School on Computer Science}{ITIS E. Mattei.}{Urbino}{\textit{100}}{}

\section{Scientific Activities}
\nocite{*}
\bibliographystyle{plain}
\bibliography{pub}

\subsection{Talks in International Conferences}

\cventry{2022}{Addressing Collective Computations Efficiency: Towards a Platform-level Reinforcement Learning Approach}{International Conference on Autonomic Computing and Self-Organizing Systems - \emph{ACSOS}}{}{}{}

\cventry{2022}{Machine learning for aggregate computing: a research roadmap}{Workshop on DIStributed COLlective Intelligence}{}{}{}

\cventry{2022}{Towards reinforcement learning-based aggregate computing}{International Conference on Coordination Models and Languages - \emph{COORDINATION}}{}{}{}

\cventry{2021}{Research directions for aggregate computing with machine learning}{Doctoral Symposium International Conference on Autonomic Computing and Self-Organizing Systems - \emph{ACSOS}}{}{}{}

%%%%%%% 
\subsection{Participation in International Conferences}
\cventry{2021}{ScaFi-Web: A Web-Based Application for Field-Based Coordination Programming}{International Conference on Coordination Models and Languages - \emph{COORDINATION}}{}{}{}
\cventry{2023}{Artifact Evaluation Committee}{International Conference on Pervasive Computing and Communications - \emph{PerCom}}{}{}{}
\cventry{2023}{Program Chair Committe}{Workshop on DIStributed COLlective Intelligence - \emph {DISCOLI}}{}{}{}
\cventry{2022}{Artifact Evaluation Committee}{International Conference on Coordination Models and Languages - \emph{DisCoTec}}{}{}{}

\cventry{2021}{Artifact Evaluation Committee}{International Conference on Autonomic Computing and Self-Organizing Systems - \emph{ACSOS}}{}{}{}

\subsection{Volunteering}
\cventry{2022}{Student Volunteer}{International Conference on Distributed Computing Systems - \emph{ICDCS}}{}{}{}
\subsection{Visiting}
\cventry{2023}{Visiting PhD}{Aarhus University -- Lukas Esterle}{Aarhus, Denmark}{}{}
\subsection{Review Activity}
\cventry{}{}{}{}{}{Reviewer for several scientific journals -- Science of Computer Programming, Scientific Programming, Frontiers in Robotics and AI}
\subsection{Research Group Collaboration}
\cventry{2021 - }{Università di Bologna}{Prof. Mirko Viroli}{In Prof. Viroli's research group, my activities have mainly focused on the topics of aggregate computing and multi-agent reinforcement learning applied to cyber swarms systems.}{}{}
\cventry{2021 - }{Università di Torino}{Prof. Ferruccio Damiani}{
  In Ferruccio Damini's group, our primary focus was on the application of aggregate computing in swarm robotics. This fruitful collaboration resulted in the publication of the paper titled ``A field-based computing approach for sensing-driven clustering in robot swarms.''}{}{}
\cventry{2021}{St. Gallen University}{Prof. Guido Salvaneschi}{In collaboration with Guido Salvaneschi, we endeavoured to expand the concepts of pulverized architecture through multitier programming languages. Our joint efforts culminated in the publication of the paper titled ``Towards Pulverized Architectures for Collective Adaptive Systems through Multi-tier Programming''}{}{}
\cventry{2022}{Aarhus Universitat}{Prof. Lukas Esterle}{Throughout my time abroad, our research was centred around exploring distributed collective intelligence within the realm of large-scale systems. Our primary emphasis was on the application of graph neural networks for developing distributed controllers.}{}{}
\section{Teaching}

\subsection{Tutoring}
\cventry{2023--today}{Concurrent and Distributed Programming}%
 {Alma Mater Studiorum -- Università di Bologna}{Master in Computer Science and Engineering}
 {}{}
\cventry{2023--today}{Programming and Development Paradigms}%
 {Alma Mater Studiorum -- Università di Bologna}{Master in Computer Science and Engineering}
 {}{}
\cventry{2022}{Concurrent and Distributed Programming}%
 {Alma Mater Studiorum -- Università di Bologna}{Master in Computer Science and Engineering}
 {}{}
\cventry{2022}{Programming and Development Paradigms}%
 {Alma Mater Studiorum -- Università di Bologna}{Master in Computer Science and Engineering}
 {}{}
\cventry{2018 - 2019}{Snap! courses}{CRIAD Coding}{Grade schools}{}{}
\subsection{Thesis (Co)Supervisor}
\cventry{2023}{Master Thesis}{Alma Mater Studiorum -- Università di Bologna}{Daily Medical Team Briefings in Ambiente Collaborativo con Schermi Multi-Touch}{}{Student: Bazzocchi, Luca}
\cventry{2023}{Master Thesis}{Alma Mater Studiorum -- Università di Bologna}{Gestione degli effetti in linguaggi di programmazione funzionale: tecniche di modellazione e interpretazione}{}{Student: Cavalieri, Giacomo}
\cventry{2022}{Bachelor Thesis}{Alma Mater Studiorum -- Università di Bologna}{Progettazione di un ambiente di programmazione visuale block-based per ScaFi. }{}{Student: Cerioni, Matteo}
\cventry{2022}{Bachelor Thesis}{Alma Mater Studiorum -- Università di Bologna}{ScaFi: Integration and Performance Analysis with Scala Native.}{}{Student: Mancini, Kevin}

\subsection{Talks}
\cventry{2023}{Intro to Deep Reinforcement Learning}{Università Di Urbino}{Talk @ Fundamentals of Artifical Intellingence}{}{}
\cventry{2022}{Engineering Cyber-Physical Swarm}{Aarhus Universitat}{Talk @ DIGIT lunch meetings}{}{}%
\cventry{2022}{Multi-Agent Reinforcement Learning, Introduction}{Alma Mater Studiorum -- Università di}{Talk @ Pervasice Computing}{}{}%
\cventry{2022}{Scala to the large}%
{Alma Mater Studiorum -- Università di Bologna}{Talk @ Programming and Development Paradigms}
{}{}
\cventry{2022}{Cross Platform in Scala}%
{Alma Mater Studiorum -- Università di Bologna}{Talk @ Programming and Development Paradigms}
{}{}

\cventry{2021}{On Collective Reinforcement Learning}%
{Alma Mater Studiorum -- Università di Bologna}{Talk @ Pervasive Computing}
{}{}

\cventry{2021}{MVC meets Monad}%
{Alma Mater Studiorum -- Università di Bologna}{Talk @ Programming and Development Paradigms}
{}{}

\cventry{2019}{Crea il tuo videogioco in Snap!}%
{Talk @ Salone dell'Orientamento, Forlì}{}
{}{}
\section{Awards}
\cventry{2023}{Best Master Thesis}{Sergio Focardi Awards}{\emph{https://www.serinar.unibo.it/gianluca-aguzzi-si-aggiudica-la-ii-edizione-del-premio-di-laurea-sergio-focardi/}}{}{}
\cventry{2017}{Prize for Meritous Students}{Alma Mater Studiorum -- Università di Bologna, Campus Cesena}{}{}{}
%\subsection{Student Supervisor}

\section{Technical Skills}

\subsection{Programming Languages}
\begin{cvcolumns}
  \cvcolumn{}{
    \cvitem{\languageknowledge{Scala}{5}}{}
    \cvitem{\languageknowledge{Java}{4}}{}
    \cvitem{\languageknowledge{C\#}{3}}{}
    \cvitem{\languageknowledge{C}{2}}{}
  }
  \cvcolumn{}{
    \cvitem{\languageknowledge{Kotlin}{3}}{}
    \cvitem{\languageknowledge{JavaScript}{3}}{}
    \cvitem{\languageknowledge{Bash}{2}}{}
    \cvitem{\languageknowledge{Prolog}{1}}{}
  }
  \cvcolumn{}{
    \cvitem{\languageknowledge{TypeScript}{1}}{}
    \cvitem{\languageknowledge{Haskell}{1}}{}
    \cvitem{\languageknowledge{C++}{2}}{}
  }
\end{cvcolumns}

\subsection{Other Languages}
\begin{cvcolumns}
  \cvcolumn{}{
    \cvitem{\languageknowledge{HTML}{3}}{}
    \cvitem{\languageknowledge{XML}{2}}{}
    \cvitem{\languageknowledge{JSON}{3}}{}
    \cvitem{\languageknowledge{RDF}{1}}{}{}
    }
  \cvcolumn{}{
    \cvitem{\languageknowledge{Markdown}{3}}{}
    \cvitem{\languageknowledge{LaTeX}{4}}{}
    \cvitem{\languageknowledge{OWL}{1}}{}
  }
  \cvcolumn{}{
    \cvitem{\languageknowledge{SPARQL}{1}}{}
    \cvitem{\languageknowledge{YAML}{2}}{}
    \cvitem{\languageknowledge{SQL}{2}}{}
  }
\end{cvcolumns}
\subsection{Libraries}
\begin{cvcolumns}
  \cvcolumn{}{
    \cvitem{\languageknowledge{Scala.js}{3}}{}
    \cvitem{\languageknowledge{Tensorflow}{2}}{}
    \cvitem{\languageknowledge{Pytorch}{1}}{}
    \cvitem{\languageknowledge{OpenAI Gym}{2}}{}
    }
  \cvcolumn{}{
    \cvitem{\languageknowledge{Monix}{4}}{}
    \cvitem{\languageknowledge{Matplotlib}{2}}{}
    \cvitem{\languageknowledge{Akka}{2}}{}
  }
  \cvcolumn{}{
    \cvitem{\languageknowledge{ScalaPy}{3}}{}
    \cvitem{\languageknowledge{Cats}{3}}{}
    \cvitem{\languageknowledge{ZIO}{1}}{}{}
  }
\end{cvcolumns}
\subsection{Software Tools}
\begin{cvcolumns}
  \cvcolumn{}{
    \cvitem{\languageknowledge{Gimp}{2}}{}{}
    \cvitem{\languageknowledge{Git}{4}}{}{}
    \cvitem{\languageknowledge{GHA}{2}}{}{}
    \cvitem{\languageknowledge{Docker}{2}}{}{}
    }
  \cvcolumn{}{
    \cvitem{\languageknowledge{Inkscape}{2}}{}{}
    \cvitem{\languageknowledge{Blender}{2}}{}{}
    \cvitem{\languageknowledge{OWL}{1}}{}{}
    \cvitem{\languageknowledge{Kdenlive}{1}}{}{}
  }
  \cvcolumn{}{
    \cvitem{\languageknowledge{NPM}{1}}{}{}
    \cvitem{\languageknowledge{SBT}{4}}{}{}
    \cvitem{\languageknowledge{Hugo}{1}}{}{}
    \cvitem{\languageknowledge{Gradle}{2}}{}{}
  }
\end{cvcolumns}
\subsection{Software Projects}
\cventry{2021 -- today}{Co-designer and main contributor of ScaFi-Web}{It is a web-based application allowing in-browser editing, execution, and visualisation of ScaFi programs.}{}{}{\url{https://github.com/scafi/scafi-web}}
\cventry{2021 -- today}{Designer of scalapy-gym}{It is a Scala facade that enable the usage of open ai gyms in the JVM!}{}{}{\url{https://github.com/cric96/scalapy-gym}}
\cventry{2020 -- today}{Co-designer of Fluvium}{An IoT project for river controll that uses AWS lambda}{}{}{\url{https://github.com/sbricco-house/fluvium}}
\subsection{Open Source Contributions}
\cventry{2018 -- today}{Development of GUI \& simulator for ScaFi}{}{}{}{\url{https://github.com/scafi/scafi}}
\cventry{2021 -- today}{Contributions to ScaFi incarnations in Alchemist}{}{}{}{\url{https://github.com/AlchemistSimulator/Alchemist}}


\section{Miscellaneous}
\cventry{2022}{Attending @ MOOC}{Introduction to Complexity @ Santa Fe Institute}{}{}{}
\cventry{2020}{Student class representative @ Alma Mater Studiorum}{}{}{}{}
\cventry{2018}{Presenting Snap! @ Notte dei ricercatori}{}{}{}{}
\cventry{2013-2015}{Student class representative @ ITIS}{}{}{}{}

%\section{References}

% Publications from a BibTeX file without multibib
%  for numerical labels: \renewcommand{\bibliographyitemlabel}{\@biblabel{\arabic{enumiv}}}% CONSIDER MERGING WITH PREAMBLE PART
%  to redefine the heading string ("Publications"): \renewcommand{\refname}{Articles}

% Publications from a BibTeX file using the multibib package
%\section{Publications}
%\nocitebook{book1,book2}
%\bibliographystylebook{plain}
%\bibliographybook{publications}                   % 'publications' is the name of a BibTeX file
%\nocitemisc{misc1,misc2,misc3}
%\bibliographystylemisc{plain}
%\bibliographymisc{publications}                   % 'publications' is the name of a BibTeX file

\clearpage

%\clearpage\end{CJK*}                              % if you are typesetting your resume in Chinese using CJK; the \clearpage is required for fancyhdr to work correctly with CJK, though it kills the page numbering by making \lastpage undefined
\end{document}


%% end of file `template.tex'.