%% start of file `template.tex'.
%% Copyright 2006-2015 Xavier Danaux (xdanaux@gmail.com).
%
% This work may be distributed and/or modified under the
% conditions of the LaTeX Project Public License version 1.3c,
% available at http://www.latex-project.org/lppl/.


\documentclass[10pt,a4paper,roman]{moderncv}        % possible options include font size ('10pt', '11pt' and '12pt'), paper size ('a4paper', 'letterpaper', 'a5paper', 'legalpaper', 'executivepaper' and 'landscape') and font family ('sans' and 'roman')

% moderncv themes
\moderncvstyle{banking}                             % style options are 'casual' (default), 'classic', 'banking', 'oldstyle' and 'fancy'
\moderncvcolor{burgundy}                               % color options 'black', 'blue' (default), 'burgundy', 'green', 'grey', 'orange', 'purple' and 'red'
%\renewcommand{\familydefault}{\sfdefault}         % to set the default font; use '\sfdefault' for the default sans serif font, '\rmdefault' for the default roman one, or any tex font name
%\nopagenumbers{}                                  % uncomment to suppress automatic page numbering for CVs longer than one page
% character encoding
%\usepackage[utf8]{inputenc}                       % if you are not using xelatex ou lualatex, replace by the encoding you are using
%\usepackage{CJKutf8}                              % if you need to use CJK to typeset your resume in Chinese, Japanese or Korean

% adjust the page margins
\usepackage[scale=0.75]{geometry}
%\setlength{\hintscolumnwidth}{3cm}                % if you want to change the width of the column with the dates
%\setlength{\makecvheadnamewidth}{10cm}            % for the 'classic' style, if you want to force the width allocated to your name and avoid line breaks. be careful though, the length is normally calculated to avoid any overlap with your personal info; use this at your own typographical risks...
% setting the colours according to \moderncvcolor
\colorlet{languagecolor}{color1}
\colorlet{nolanguagecolor}{color2}
\newcount\languagecount
\newcommand\languageknowledge[2]
  {%
    % if you change the 4cm you can change the distance, the 4cm is the space
    % reserved for the language's name, immediately after it the dots are
    % printed.
    %\hfill % if you want the dots to be at the right margin
    \languagecount=0
    \loop\ifnum\languagecount<#2
      \advance\languagecount1
      \textcolor{languagecolor}{$\bullet$}%
    \repeat
    \loop\ifnum\languagecount<5
      \advance\languagecount1
      \textcolor{nolanguagecolor}{$\bullet$}%
    \repeat
    \makebox[3cm][l]{ \textbf{#1}}%
  }

% personal data
\name{Gianluca}{Aguzzi}
\title{\\Post Doctoral Researcher}
\address{Via Mulini 23/25}{47521}{Italy, Cesena}
\email{gianluca.aguzzi@unibo.it}
\homepage{https://cric96.github.io/}
\social[orcid]{0000-0002-1553-4561}
\social[github]{cric96}
\social[stackoverflow]{gianluca-aguzzi}
% bibliography adjustements (only useful if you make citations in your resume, or print a list of publications using BibTeX)
%   to show numerical labels in the bibliography (default is to show no labels)
%\makeatletter\renewcommand*{\bibliographyitemlabel}{\@biblabel{\arabic{enumiv}}}\makeatother
\renewcommand*{\bibliographyitemlabel}{[\arabic{enumiv}]}
%   to redefine the bibliography heading string ("Publications")
%\renewcommand{\refname}{Articles}

% bibliography with mutiple entries
\usepackage{multibib}
\newcites{book,misc}{{Books},{Others}}
%----------------------------------------------------------------------------------
%            content
%----------------------------------------------------------------------------------
\begin{document}

\makecvtitle

\subsection{Current Placement}
\cventry{2023--today}{Postdoctoral Research}%
 {Alma Mater Studiorum -- University of Bologna}{Cesena}
 {}{I'm currently work on a PRIN project called COMMONS-WEARS in which I deal with engineering collective application in complex layered and multi mobile edge devices}
\subsection{Research Theme}
\cvitem{}{
  My current area of research focuses on the coordination of large-scale systems and the application of machine learning algorithms within distributed and multi-agent systems.
  Specifically, my interests lie in the \textbf{design} and \textbf{implementation} of self-adaptive systems through the use of advanced \emph{modeling techniques}, \emph{programming languages}, and \emph{methodologies}.
  Additionally, I explore the integration of machine learning algorithms, particularly \emph{many-agent reinforcement learning}, to enhance the \textbf{effectiveness}, \textbf{efficiency}, and \textbf{intelligence} of these collective programs.
}
\section{Education}

%%%%%
\cventry{2020--2023}{PhD in  Computer Science and Software Engineering}%
 {Alma Mater Studiorum -- University of Bologna}{Cesena}
 {\textit{with distinction}}{
 My PhD was focused on engineering large scale systems through the application of aggregate computing and machine learning algorithms. In particular, I explored the application of multi-agent reinforcement learning in the context of cyber-physical swarms -- a type of large-scale system composed of multiple agents that interact with the physical world.
 }
\cvitem{}{Thesis: \emph{A language-based software engineering approach for cyber-physical swarms}}
\cvitem{}{Supervisors: \emph{Mirko Viroli}}
\cventry{2018--2020}{Master in  Computer Science and Software Engineering}%
 {Alma Mater Studiorum -- University of Bologna}{Cesena}
 {\textit{110 cum Laude}}{
  In this master I focused on the study of programming languages, software engineering, and distributed systems. I also explored the application of aggregate computing in the development of large-scale systems.
  Moreover, I developed a strong interest in the application of machine learning algorithms in the context of distributed systems, particularly in the field of multi-agent reinforcement learning.
 }
\cvitem{}{Thesis: \emph{Scafi web: a Scala-JavaScript platform for executing, simulating, and controlling aggregate computing systems}}
\cvitem{}{Supervisors: \emph{Mirko Viroli, Roberto Casadei}}
%%%%
\cventry{2015--2018}{Bachelor in Computer Science and Software Engineering}{Alma Mater Studiorum -- University of Bologna}{Cesena}{\textit{110 cum Laude}}{}

\cvitem{}{Thesis: \emph{Sviluppo di un front-end di simulazione per applicazioni aggregate nel framework Scafi}}
\cvitem{}{Supervisors: \emph{Mirko Viroli, Roberto Casadei}}

\cventry{2015--2018}{High School on Computer Science}{ITIS E. Mattei.}{Urbino}{\textit{100}}{}

\section{Scientific Activities}
\nocite{*}
\bibliographystyle{plainyr-rev}
\bibliography{pub}

\subsection{Presentations in International Conferences}
\cventry{}{\textbf{COORDINATION 2024}}{}{}{}{Scafi-blocks: A visual aggregate programming environment for low-code swarm design}
\cventry{}{\textbf{DISCOLI 2024}}{}{}{}{Engineering distributed collective intelligence in cyber-
physical swarms}
\cventry{}{\textbf{ACSOS 2023}}{}{}{}{Field-informed Reinforcement Learning of Collective Tasks with Graph Neural Networks}
\cventry{}{\textbf{Scala Italy  2023}}{}{}{}{Scala in machine learning scenario: a personal experience}
\cventry{}{\textbf{COORDINATION 2023}}{}{}{}{Macroswarm: A field-based compositional framework for swarm programming}
\cventry{}{\textbf{COORDINATION 2023}}{}{}{}{Scarlib: A framework for cooperative many agent deep reinforcement learning in Scala}
\cventry{}{\textbf{ACSOS 2022}}{}{}{}{Addressing Collective Computations Efficiency: Towards a Platform-level Reinforcement Learning Approach}
\cventry{}{\textbf{DISCOLI 2023}}{}{}{}{Machine learning for aggregate computing: a research roadmap}
\cventry{}{\textbf{COORDINATION 2022}}{}{}{}{Towards reinforcement learning-based aggregate computing}
\cventry{}{\textbf{Doctoral Symposium International @ ACSOS 2021}}{}{}{}{Research directions for aggregate computing with machine learning}
\cventry{}{\textbf{COORDINATION 2021}}{}{}{}{ScaFi-Web: A Web-Based Application for Field-Based Coordination Programming}

%%%%%%% 
\subsection{Participation in International Conferences}
\cventry{2024}{Demo and Poster Committee}{Autonomic Computing and Self-Organizing Systems -- ACSOS}{}{}{}
\cventry{2024}{Organising Chair Committe}{Workshop on DIStributed COLlective Intelligence -- DISCOLI}{}{}{}
\cventry{2024}{Program Chair Committe}{MADTECC}{}{}{}
\cventry{2023}{Artifact Evaluation Committee}{International Conference on Pervasive Computing and Communications - \emph{PerCom}}{}{}{}
\cventry{2023}{Program Chair Committe}{Workshop on DIStributed COLlective Intelligence - \emph {DISCOLI}}{}{}{}
\cventry{2022}{Artifact Evaluation Committee}{International Conference on Coordination Models and Languages - \emph{DisCoTec}}{}{}{}

\cventry{2021}{Artifact Evaluation Committee}{International Conference on Autonomic Computing and Self-Organizing Systems - \emph{ACSOS}}{}{}{}

\subsection{Volunteering}

\cventry{2022}{Student Volunteer}{International Conference on Distributed Computing Systems - \emph{ICDCS}}{}{}{}
\cventry{2022}{Student Volunteer}{Internation Conference on Autonomic Computing and Self-Organising Systems - ACSOS}{}{}{}

\subsection{Visiting}
\cventry{2023}{Visiting PhD}{Aarhus University -- Lukas Esterle}{Aarhus, Denmark}{}{}
\subsection{Review Activity}
\cventry{}{\textbf{Reviewer for several scientific journals}}{}{}{}{ Science of Computer Programming, Scientific Programming, Frontiers in Robotics and AI, Hindawi, Autonomous Agents and Multi-Agent Systems}

\cventry{}{\textbf{Reviewer for international conferences and Workshop}}{}{}{}{ COORDINATION, ACSOS, DISCOLI, AAMAS, PerCom, ASE NIER}

\subsection{Research Group Collaboration}
\cventry{2021 - }{University of Bologna}{Prof. Mirko Viroli}{}{}{In Prof. Viroli's research group, my activities have mainly focused on the topics of aggregate computing and multi-agent reinforcement learning applied to cyber swarms systems.}
\cventry{2021 - }{University of Turin}{Prof. Ferruccio Damiani}{}{}{
  In Ferruccio Damini's group, our primary focus was on the application of aggregate computing in swarm robotics. This fruitful collaboration resulted in the publication of the paper titled ``A field-based computing approach for sensing-driven clustering in robot swarms.''}
\cventry{2021}{St. Gallen University}{Prof. Guido Salvaneschi}{}{}{In collaboration with Guido Salvaneschi, we endeavoured to expand the concepts of pulverized architecture through multitier programming languages. Our joint efforts culminated in the publication of the paper titled ``Towards Pulverized Architectures for Collective Adaptive Systems through Multi-tier Programming''}
\cventry{2022 -}{Aarhus Universitat}{Prof. Lukas Esterle}{}{}{Throughout my time abroad, our research was centred around exploring distributed collective intelligence within the realm of large-scale systems. Our primary emphasis was on the application of graph neural networks for developing distributed controllers.}

\subsection{PhD Schools}
\cventry{2023}{PhD Summer School}{10$^{th}$ DeepLearn Summer School}{}{}{}
\cventry{2021}{PhD Summer School}{22$^{nd}$ European Agent Systems Summer School}{}{}{}

\section{Teaching}
\subsection{Courses}
\cventry{2023 - today}{Software Design and Development}{Alma Mater Studiorum -- University of Bologna}{Bachelor in Computer Science and Engineering}{}{
    In this course, we explore the principles of software design and development, focusing on the application of object-oriented programming and design patterns.
  }{}
\cventry{2023}{Introduction to Reinforcement Learning -- 4 hours}%
 {Advanced School in Artificial Intelligence}{\url{https://asai-er.github.io/services/docenti/}}
 {}{}
\subsection{Tutoring}
\cventry{2022--today}{Concurrent and Distributed Programming}%
 {Alma Mater Studiorum -- University of Bologna}{Master in Computer Science and Engineering}
 {}{}
\cventry{2022--today}{Programming and Development Paradigms}%
 {Alma Mater Studiorum -- University of Bologna}{Master in Computer Science and Engineering}
 {}{}

\cventry{2018 - 2019}{Snap! courses}{CRIAD Coding}{Grade schools}{}{}
\subsection{Thesis (Co)Supervisor - Selected}
\cventry{2024}{Master Thesis}{}{}{Student: Luce Deluigi}{\textbf{Design and implementation of a scalable domain specific language foundation for ScaFi with Scala 3.}}
\cventry{2024}{Master Thesis}{}{}{Student: Davide Domini}{\textbf{Aggregate Computing and Many-Agent Reinforcement Learning: Towards a Hybrid Toolchain}}
\cventry{2023}{Master Thesis}{}{}{Student: Francesco Dente}{\textbf{A functional-reactive perspective on the Aggregate Computing paradigm}}
\cventry{2023}{Master Thesis}{}{}{Student: Giacomo Cavalieri}{\textbf{Gestione degli effetti in linguaggi di programmazione funzionale: tecniche di modellazione e interpretazione}}
\cventry{2022}{Bachelor Thesis}{}{}{Student: Cerioni, Matteo}{\textbf{Progettazione di un ambiente di programmazione visuale block-based per ScaFi.}}
\cventry{2022}{Bachelor Thesis}{}{}{Student: Mancini, Kevin}{\textbf{ScaFi: Integration and Performance Analysis with Scala Native.}}
\cventry{}{}{}{}{}{For the complete list of supervised thesis, please visit AMS thesis: \url{https://amslaurea.unibo.it/view/relatore/Aguzzi=3AGianluca=3A=3A/}}{}
\subsection{Talks}
\cventry{2023}{\textbf{Multi-Agent Reinforcement Learning - Introduction}}{}{}{}{Advanced Software Modelling and Design}
\cventry{2024}{\textbf{Deep Reinforcement Learning -- Introduction}}{}{}{}
{Fundamentals of Artifical Intellingence - University of Urbino}
\cventry{2024}{\textbf{It’s all about effects - Effect systems in Functional Programming}}{}{}{}{Advanced Software Modelling and Design}
\cventry{2024}{\textbf{Leveraging Large Language Models in Software Engineering}}{}{}{}{Advanced Software Modelling and Design}
\cventry{2023}{\textbf{Multi-Agent Reinforcement Learning, Unleashing Collective Intelligence}}{}{}{}{Advanced School in Artificial Intelligence Summer School}
\cventry{2023}{\textbf{Intro to Deep Reinforcement Learning}}{}{}{}
{Fundamentals of Artifical Intellingence - University of Urbino}
\cventry{2022}{\textbf{Engineering Cyber-Physical Swarm}}{}{}{}
{DIGIT lunch meetings -- Aarhus Universitat}%
\cventry{2022}{\textbf{Multi-Agent Reinforcement Learning, Introduction}}{}{}{}{Talk @ Pervasive Computing - University of Bologna}%

\cventry{2022}{\textbf{Scala to the large}}{}{}{}%
{Programming and Development Paradigms - University of Bologna}

\cventry{2022}{\textbf{Cross Platform in Scala}}{}{}{}
{Programming and Development Paradigms - University of Bologna}

\cventry{2021}{\textbf{On Collective Reinforcement Learning}}{}{}{}
{Alma Mater Studiorum -- University of Bologna}{Talk @ Pervasive Computing}

\cventry{2021}{\textbf{MVC meets Monad}}{}{}{}
{Programming and Development Paradigms -- University of Bologna}

\cventry{2019}{\textbf{Create your own video game in Snap!}}{}{}{}
{Orientation Fair - Forlì}

\section{Awards}
\cventry{2023}{Best Master Thesis}{Sergio Focardi Awards}{}{}{\emph{https://www.serinar.unibo.it/gianluca-aguzzi-si-aggiudica-la-ii-edizione-del-premio-di-laurea-sergio-focardi/}}
\cventry{2017}{Prize for Meritous Students}{Alma Mater Studiorum -- University of Bologna, Campus Cesena}{}{}{}
%\subsection{Student Supervisor}

\section{Technical Skills}

\subsection{Programming Languages}
\begin{cvcolumns}
  \cvcolumn{}{
    \cvitem{\languageknowledge{Scala}{5}}{}
    \cvitem{\languageknowledge{Java}{4}}{}
    \cvitem{\languageknowledge{C\#}{3}}{}
    \cvitem{\languageknowledge{C}{2}}{}
  }
  \cvcolumn{}{
    \cvitem{\languageknowledge{Kotlin}{3}}{}
    \cvitem{\languageknowledge{JavaScript}{3}}{}
    \cvitem{\languageknowledge{Bash}{2}}{}
    \cvitem{\languageknowledge{Prolog}{1}}{}
  }
  \cvcolumn{}{
    \cvitem{\languageknowledge{TypeScript}{1}}{}
    \cvitem{\languageknowledge{Haskell}{1}}{}
    \cvitem{\languageknowledge{C++}{2}}{}
  }
\end{cvcolumns}

\subsection{Other Languages}
\begin{cvcolumns}
  \cvcolumn{}{
    \cvitem{\languageknowledge{HTML}{3}}{}
    \cvitem{\languageknowledge{XML}{2}}{}
    \cvitem{\languageknowledge{JSON}{3}}{}
    \cvitem{\languageknowledge{RDF}{1}}{}{}
    }
  \cvcolumn{}{
    \cvitem{\languageknowledge{Markdown}{3}}{}
    \cvitem{\languageknowledge{LaTeX}{4}}{}
    \cvitem{\languageknowledge{OWL}{1}}{}
  }
  \cvcolumn{}{
    \cvitem{\languageknowledge{SPARQL}{1}}{}
    \cvitem{\languageknowledge{YAML}{2}}{}
    \cvitem{\languageknowledge{SQL}{2}}{}
  }
\end{cvcolumns}
\subsection{Libraries}
\begin{cvcolumns}
  \cvcolumn{}{
    \cvitem{\languageknowledge{Scala.js}{3}}{}
    \cvitem{\languageknowledge{Tensorflow}{2}}{}
    \cvitem{\languageknowledge{Pytorch}{1}}{}
    \cvitem{\languageknowledge{OpenAI Gym}{2}}{}
    }
  \cvcolumn{}{
    \cvitem{\languageknowledge{Monix}{4}}{}
    \cvitem{\languageknowledge{Matplotlib}{2}}{}
    \cvitem{\languageknowledge{Akka}{2}}{}
  }
  \cvcolumn{}{
    \cvitem{\languageknowledge{ScalaPy}{3}}{}
    \cvitem{\languageknowledge{Cats}{3}}{}
    \cvitem{\languageknowledge{ZIO}{1}}{}{}
  }
\end{cvcolumns}
\subsection{Software Tools}
\begin{cvcolumns}
  \cvcolumn{}{
    \cvitem{\languageknowledge{Gimp}{2}}{}{}
    \cvitem{\languageknowledge{Git}{4}}{}{}
    \cvitem{\languageknowledge{GHA}{2}}{}{}
    \cvitem{\languageknowledge{Docker}{2}}{}{}
    }
  \cvcolumn{}{
    \cvitem{\languageknowledge{Inkscape}{2}}{}{}
    \cvitem{\languageknowledge{Blender}{2}}{}{}
    \cvitem{\languageknowledge{OWL}{1}}{}{}
    \cvitem{\languageknowledge{Kdenlive}{1}}{}{}
  }
  \cvcolumn{}{
    \cvitem{\languageknowledge{NPM}{1}}{}{}
    \cvitem{\languageknowledge{SBT}{4}}{}{}
    \cvitem{\languageknowledge{Hugo}{1}}{}{}
    \cvitem{\languageknowledge{Gradle}{2}}{}{}
  }
\end{cvcolumns}
\subsection{Software Projects}
\cventry{2023 -- today}{It is a field-based compositional framework for swarm programming.}{Designer of Macroswarm}{}{}{\url{https://github.com/scafi/macro-swarm}}
\cventry{2023 -- todat}{It is a framework for reactive self-organizing programming}{CO-deisigner of FRASP}{}{}{\url{https://github.com/cric96/distributed-frp}}
\cventry{2023 -- today}{It is a framework for cooperative many agent deep reinforcement learning in Scala}{Designer of Scarlib}{}{}{\url{https://github.com/ScaRLib-group/ScaRLib}}
\cventry{2021 -- today}{It is a web-based application allowing in-browser editing and execution of ScaFi programs.}{Co-designer and main contributor of ScaFi-Web}{}{}{\url{https://github.com/scafi/scafi-web}}
\cventry{2021 -- today}{It is a Scala facade that enable the usage of open ai gyms in the JVM!}{Designer of scalapy-gym}{}{}{\url{https://github.com/cric96/scalapy-gym}}
%\cventry{2020 -- today}{Co-designer of Fluvium}{An IoT project for river controll that uses AWS lambda}{}{}{\url{https://github.com/sbricco-house/fluvium}}
\subsection{Open Source Contributions}
\cventry{2018 -- today}{Development of GUI \& simulator for ScaFi}{}{}{}{\url{https://github.com/scafi/scafi}}
\cventry{2021 -- today}{Contributions to ScaFi incarnations in Alchemist}{}{}{}{\url{https://github.com/AlchemistSimulator/Alchemist}}


\section{Miscellaneous}
\cventry{2020}{\textbf{Student class representative @ Alma Mater Studiorum}}{}{}{}{}
\cventry{2018}{\textbf{Presenting Snap! @ Researcher Night}}{}{}{}{The researcher night is an national event that aims to bring the research to the public. In that occasion I presented Snap! to the public as a tool for teaching the computational thinking to the youngest.}
%\cventry{2013-2015}{Student class representative @ ITIS}{}{}{}{}

%\section{References}

% Publications from a BibTeX file without multibib
%  for numerical labels: \renewcommand{\bibliographyitemlabel}{\@biblabel{\arabic{enumiv}}}% CONSIDER MERGING WITH PREAMBLE PART
%  to redefine the heading string ("Publications"): \renewcommand{\refname}{Articles}

% Publications from a BibTeX file using the multibib package
%\section{Publications}
%\nocitebook{book1,book2}
%\bibliographystylebook{plain}
%\bibliographybook{publications}                   % 'publications' is the name of a BibTeX file
%\nocitemisc{misc1,misc2,misc3}
%\bibliographystylemisc{plain}
%\bibliographymisc{publications}                   % 'publications' is the name of a BibTeX file

\clearpage

%\clearpage\end{CJK*}                              % if you are typesetting your resume in Chinese using CJK; the \clearpage is required for fancyhdr to work correctly with CJK, though it kills the page numbering by making \lastpage undefined
\end{document}


%% end of file `template.tex'.